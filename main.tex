%%%%%%%%%%%%%%%%%%%%%%%%%%%%%%%%%%%%%%%%%%%%%%%%%%%%%%%%%%%%%%%%%%%%%%%%%%%%%%%%%%%%%%%%%%%%%%%%%%%%%%%%%%%%%%%%%
%                                     PAPER ROMMEL - ELSY                                                       %
%%%%%%%%%%%%%%%%%%%%%%%%%%%%%%%%%%%%%%%%%%%%%%%%%%%%%%%%%%%%%%%%%%%%%%%%%%%%%%%%%%%%%%%%%%%%%%%%%%%%%%%%%%%%%%%%%

%%%%%%%%%%%%%%%%%%%%%%%% DECLARACIONES DE LIBRERIAS Y PAQUETES %%%%%%%%%%%%%%%%%%%%%%%%%%%%%%%%%%%%%%%%%%%%%%%%%
\documentclass[conference]{IEEEtran}
\IEEEoverridecommandlockouts
% The preceding line is only needed to identify funding in the first footnote. If that is unneeded, please comment it out.
\usepackage{cite} %paquete de la cita bibliografica
\usepackage{amsmath}
\usepackage{amsmath,amssymb,amsfonts} %paquete de fuente
\usepackage{algorithmic} %paquete de simbología númerica
\usepackage{graphicx} %paquete de imagenes
\usepackage{textcomp} %paquete de 
\usepackage{xcolor} %paquete de los colores en texto
\usepackage{hyperref} %paquete de link en el texto
\def\BibTeX{{\rm B\kern-.05em{\sc i\kern-.025em b}\kern-.08em
    T\kern-.1667em\lower.7ex\hbox{E}\kern-.125emX}} %paquete de estructura de las cita bibliografica
\definecolor{orcidlogocol}{HTML}{A6CE39} %paquete de un color especifico para el logo del codigo orcid.
\usepackage{flafter}
\usepackage{amsmath}
\usepackage{multirow}
\usepackage{colortbl}
\usepackage{tablefootnote}
\usepackage{booktabs}
%%%%%%%%%%%%%%%%%%%%%%%%%%%%%%%%%%%%%%%%%%%%%%%%%%%%%%%%%%%%%%%%%%%%%%%%%%%%%%%%%%%%%%%%%%%%%%%%%%%%%%%%%%%%%%%%%
%                                    INICIO DEL DOCUMENTO                                                       %
%%%%%%%%%%%%%%%%%%%%%%%%%%%%%%%%%%%%%%%%%%%%%%%%%%%%%%%%%%%%%%%%%%%%%%%%%%%%%%%%%%%%%%%%%%%%%%%%%%%%%%%%%%%%%%%%%
\begin{document}
%%%%%%%%%%%%%%%%%%%%%%%%%%%%%%%%%%%%%%%%%%%%%%%%%%%%%%%%%%%%%%%%%%%%%%%%%%%%%%%%%%%%%%%%%%%%%%%%%%%%%%%%%%%%%%%%%
%                                     TITULO Y AUTORES                                                          %
%%%%%%%%%%%%%%%%%%%%%%%%%%%%%%%%%%%%%%%%%%%%%%%%%%%%%%%%%%%%%%%%%%%%%%%%%%%%%%%%%%%%%%%%%%%%%%%%%%%%%%%%%%%%%%%%%

\title{Analyze the implementation and impact of Communication-Based Train Control (CBTC) and Automatic Train Protection (ATP) railway signaling systems on the operational improvement of railway projects in Ecuador\\
%{\footnotesize \textsuperscript{*}Note: Sub-titles are not captured in Xplore and
%should not be used}
}
%%%%%%%%%%%%%%%%%%%%%%%% PRIMER AUTOR %%%%%%%%%%%%%%%%%%%%%%%%%%%%%%%%%%%%%%%%%%%%%%%%%
\author{\IEEEauthorblockN{1\textsuperscript{st} Elsy del Rocío Villamar Garcés}
\IEEEauthorblockA{\textit{Escuela Técnica Superior de Ingenieros Industriales}\\
\textit{Universidad Politecnica de Madrid}\\
Madrid, Spain \\
evillamar@alumnos.upm.es\\
\includegraphics[keepaspectratio,width=0.9em]{Imagenes_general/ORCIDiD_icon128x128.png}
\href{https://orcid.org/0000-0001-7041-8611}{{orcid.org/0000-0001-7041-8611}}
}\\
\and
%%%%%%%%%%%%%%%%%%%%%%%% SEGUNDO AUTOR %%%%%%%%%%%%%%%%%%%%%%%%%%%%%%%%%%%%%%%%%%%%%%%%%
\IEEEauthorblockN{2\textsuperscript{nd} Rommel Andrés Rivera Buenaño}
\IEEEauthorblockA{\textit{ Mechanical Engineering Department} \\
\textit{}\\
Madrid, Spain \\
\includegraphics[keepaspectratio,width=0.9em]{Imagenes_general/ORCIDiD_icon128x128.png}
\href{https://orcid.org/0009-0002-2130-5922}{{orcid.org/0009-0002-2130-5922}}
}\\
\and
%%%%%%%%%%%%%%%%%%%%%%%% TERCER AUTOR %%%%%%%%%%%%%%%%%%%%%%%%%%%%%%%%%%%%%%%%%%%%%%%%%
\IEEEauthorblockN{3\textsuperscript{rd}  Javier Gómez Fernández}
\IEEEauthorblockA{\textit{Mechanical Engineering Department} \\
\textit{Universidad Politecnica de Madrid}\\
Madrid, Spain \\
\includegraphics[keepaspectratio,width=0.9em]{Imagenes_general/ORCIDiD_icon128x128.png}
\href{https://orcid.org/0000-0002-3382-5776}{{orcid.org/0000-0002-3382-5776}}
}\\
\and
%%%%%%%%%%%%%%%%%%%%%%%% CUARTO AUTOR %%%%%%%%%%%%%%%%%%%%%%%%%%%%%%%%%%%%%%%%%%%%%%%%%
\IEEEauthorblockN{4\textsuperscript{to} Juan de Dios Sanz Bobi}
\IEEEauthorblockA{\textit{Mechanical Engineering Department} \\
\textit{Universidad Politecnica de Madrid}\\
Madrid, Spain \\
\includegraphics[keepaspectratio,width=0.9em]{Imagenes_general/ORCIDiD_icon128x128.png}
\href{https://orcid.org/0000-0002-8039-8527}{{orcid.org/0000-0002-8039-8527}}
}}
\maketitle
%%%%%%%%%%%%%%%%%%%%%%%%%%%%%%%%%%%%%%%%%%%%%%%%%%%%%%%%%%%%%%%%%%%%%%%%%%%%%%%%%%%%%%%%%%%%%%%%%%%%%%%%%%%%%%%%%
%                                    RESUMEN                                                                    %
%%%%%%%%%%%%%%%%%%%%%%%%%%%%%%%%%%%%%%%%%%%%%%%%%%%%%%%%%%%%%%%%%%%%%%%%%%%%%%%%%%%%%%%%%%%%%%%%%%%%%%%%%%%%%%%%%
\begin{abstract}
This study analyzes the implementation and impact of CBTC and ATP signaling systems in railway projects in Ecuador. ATP ensures safety through train detection, separation, and interlocking control. CBTC systems, which support several grades of automation, offer advantages in reliability, capacity, energy efficiency, and scalability compared to ATP. Using the Binary Decision Diagrams (BDD) method, the study demonstrates that CBTC outperforms ATP in terms of reliability (12.85\%), safety (10.5\%), capacity (2.8\%), maintenance (1.66\%), and costs (4.5\%). Simulations indicate that transitioning from ATP to CBTC improves capacity by optimizing headway to 30 seconds less than the theoretical value in the studied case, and enhances traffic management by adding 6 more trains to the line. The study concludes that adopting CBTC technology in Ecuador is a strategic investment with substantial long-term benefits in operational efficiency, safety, and reliability.
\end{abstract}

\begin{IEEEkeywords}
CBTC, ATP, ATO, ATS, Headway, Railway Signaling, UGT
\end{IEEEkeywords}

%%%%%%%%%%%%%%%%%%%%%%%%%%%%%%%%%%%%%%%%%%%%%%%%%%%%%%%%%%%%%%%%%%%%%%%%%%%%%%%%%%%%%%%%%%%%%%%%%%%%%%%%%%%%%
%                                    INTRODUCCIÓN                                                           %
%%%%%%%%%%%%%%%%%%%%%%%%%%%%%%%%%%%%%%%%%%%%%%%%%%%%%%%%%%%%%%%%%%%%%%%%%%%%%%%%%%%%%%%%%%%%%%%%%%%%%%%%%%%%%

\section{Introduction}
In the initial stages, railway signaling depended on human operators and telephonic communications to coordinate between stations, crucial for ensuring safety; however, with technological advancements, ATP systems have significantly enhanced protection, safety, and efficiency in railway operations.

\begin{figure}[htbp]
    \centering
    \centerline{\includegraphics[scale=0.20, angle=0]{Imagenes_general/ATP_Architecture_Metro_Quito_1.pdf}}
    \caption{ATP Architecture diagram}
    \label{ATP Architecture diagram}
\end{figure}

The Automatic Train Protection (ATP) system forms the foundation of modern railway signaling systems. It has evolved to meet the critical demands for train integrity verification and precise, real-time location tracking. This is achieved through continuous Communications-Based Train Control (CBTC) communication between trackside and onboard equipment, significantly enhancing safety and efficiency on the rails. Figure [\ref{ATP Architecture diagram}]

\begin{figure}[ht]
    \centering
    \centerline{\includegraphics[width=0.5\textwidth, scale=0.90]{Imagenes_general/esquema_general_arquitectura_CBTC_1.pdf}}
    \caption{CBTC Architecture diagram}
    \label{CBTC Architecture diagram}
\end{figure}

CBTC system supports several levels of automation, including driverless train operations, and offers advantages such as increased service reliability, energy efficiency, and system scalability. This system enables more precise train control using the moving block principle, which dynamically adjusts the safe distance between trains based on their speed and braking capacity, unlike the fixed block system, which divides the track into fixed sections for safety but limits operational capacity and train proximity. In the IEEE 1474.1-2004 standard \cite{b1}, it is more detailed. Figure [\ref{CBTC Architecture diagram}]

In Urban Guided Transport (UGT), the grade of automation ranges from manual to autonomous operation (GoA0 to GoA4 according to UNE-EN-62290-1 \cite{b17}), include intermediate levels of assistance and automatic operation, with variations in human intervention and the technology used.

The Automatic Train Protection (ATP) and the Communication-Based Train Control (CBTC) systems operate individually to constantly monitor the speed and location of trains, ensuring safe and adaptive operations in response to changes in traffic and track conditions, which minimizes human intervention and reduces the risk of accidents. ATP ensures that trains maintain safe distances, and beside this, CBTC allows operations at shorter intervals, thereby increasing service frequency and optimizing the efficiency of the existing infrastructure. The integration of these systems enhances diagnostics and problem resolution, facilitating faster maintenance and reduced costs. These benefits offset the high initial cost of implementation, resulting in significant long-term savings. This advanced configuration not only strengthens the safety and capacity of the trains but also optimizes operational efficiency, making it a strategic investment to improve operations and passenger service.

The objectives of this article:
\begin{itemize}
\item Evaluate the current capabilities of railway signaling systems in Ecuador and determine the operational needs that could be met by implementing CBTC and ATP technologies.
\item Examine the impact of adopting CBTC and ATP systems on optimizing railway headway, aiming to significantly increase the efficiency and operational capacity of the railway systems in Ecuador.
\item Provide recommendations for identifying and categorizing the technical, financial obstacles faced by the implementation of CBTC signaling system in Ecuador Urban Guided Transport systems.
\end{itemize}

%%%%%%%%%%%%%%%%%%%%%%%%%%%%%%%%%%%%%%%%%%%%%%%%%%%%%%%%%%%%%%%%%%%%%%%%%%%%%%%%%%%%%%%%%%%%%%%%%%%%%%%%%%%%%%
%                                    SECCIONES Y DESARROLLO                                                  %
%%%%%%%%%%%%%%%%%%%%%%%%%%%%%%%%%%%%%%%%%%%%%%%%%%%%%%%%%%%%%%%%%%%%%%%%%%%%%%%%%%%%%%%%%%%%%%%%%%%%%%%%%%%%%%

\section{Railway Projects in Ecuador}

In Ecuador, railway projects are increasingly taking a central role in efforts to modernize transportation infrastructure and efficiently connect various regions and cities of the country, thereby revitalizing both the local economy and access to essential services \cite{b2}. 

\subsection{Regional train}

Between 1872 and 1930, Ecuador significantly expanded its railway network, focusing on internal integration and the transport of agricultural products and passengers. However, growth ceased in 1930, leading to centralized administration and current use primarily for tourism, with notable routes including Alausí-Sibambe and Quito-El Boliche-Machachi. Figure[\ref{fig:timeline_train_reg}]
\begin{figure}[htbp]
    \centering
    \includegraphics[width=0.4\textwidth]{Imagenes_general/Timeline railway Guayaquil - Quito_1.pdf}
    \caption{Timeline Ecuadorian Regional train}
    \label{fig:timeline_train_reg}
\end{figure}

\subsection{Tramway de los Cuatro Rios}

Cuenca, renowned for its colonial architecture and vibrant urban life, initiated a tram project during the 2009-2014 period with a budget of 180 million dollars, aimed at revitalizing urban mobility and facing challenges such as delays and criticisms, yet emerging as a symbol of modernity and effective management \cite{b3}. Figure [\ref{fig:timeline_train_cu}]

\begin{figure}[htbp]
    \centering
    \includegraphics[width=0.5\textwidth, scale=0.50]{Imagenes_general/Tranvía de Cuenca_en.pdf}
    \caption{Timeline Tram Cuenca}
    \label{fig:timeline_train_cu}
\end{figure}

\subsection{Metro de Quito}

Quito, the capital of Ecuador, is advancing sustainable urban development with its inaugural Metro project, aiming to alleviate congestion through efficient and eco-friendly transportation, facing technical and financial challenges, yet promising to transform mobility and urban life. Figure [\ref{Timeline Metro Quito}]

\begin{figure}[htbp]                                     
    \centerline{\includegraphics[width=0.5\textwidth, scale=0.50]{Imagenes_general/Metro de Quito_en.pdf}}
    \caption{Timeline Metro Quito}                     
    \label{Timeline Metro Quito}                                                 
\end{figure} 

Finally, Ecuador's railway projects, such as the Regional Train, the Cuenca Tram, and the Quito Metro, represent significant advances in the country's transportation infrastructure. Specifically, the Quito Metro, with its implementation of ATP system, demonstrates a strong commitment to safety and efficiency. Adopting even more advanced systems could lead to further improvements in reliability, capacity, and safety, while simultaneously optimizing costs and maintenance. These improvements would not only strengthen the transportation network but also increase user confidence and satisfaction.

%%%%%%%%%%%%%%%%%%%%%%%%%%%%%%%%%%%%%%%%%%%%%%%%%%%%%%%%%%%%%%%%%%%%%%%%%%%%%%%%%%%%%%%%%%%%%%%%%%%%%%%%%%%%%%
%                                  Desarrollo                                                              %
%%%%%%%%%%%%%%%%%%%%%%%%%%%%%%%%%%%%%%%%%%%%%%%%%%%%%%%%%%%%%%%%%%%%%%%%%%%%%%%%%%%%%%%%%%%%%%%%%%%%%%%%%%%%%%

\section{Methodology}
The following analysis used the Binary Decision Diagram (BDD) \cite{b6} \cite{b5} method to characterize processes or components in specific variables, with the objective of evaluating ATP systems and CBTC systems in metropolitan urban transportation. This facilitates comparative assessments and observations of the following characteristics: reliability\cite{b7}, safety, capacity, maintenance and costs through graphical representation.
\begin{figure}[htbp]
 \centering
        \includegraphics[width=0.25\textwidth,scale=1]{Imagenes_general/Gráfico_Reliability.jpg}
        \includegraphics[width=0.20\textwidth,scale=1]{Imagenes_general/Rliability_tabla.jpg}
\caption{Reliability Analysis}
 \label{fig:Reliability Analysis}
\end{figure}

\subsection{Reliability}
The reliability of ATP relies on components like balises and speed controllers for flexible operation, while CBTC requires flawless execution of sophisticated systems like positioning and wireless communication to ensure exceptional adaptability and reliability, with both systems emphasizing the critical role of effective communication in smooth railway operations. Figure [\ref{fig:Reliability Analysis}]\\
\begin{figure}[htbp]
    \centering
    \includegraphics[width=0.28\textwidth,scale=1]{Imagenes_general/Gráfico_Safety.jpg}
    \includegraphics[width=0.13\textwidth,scale=1]{Imagenes_general/Tabla_Safety.png}
    \caption{Safety Analysis}
    \label{fig:Safety Analysis}
\end{figure}
\subsection{Safety}
ATP systems rely on traditional components and driver responses for safety, whereas CBTC systems use advanced positioning and communication technologies for more autonomous operation, with CBTC also excelling in adaptability and scalability by integrating technological advancements to enhance safety and reliability. Figure [\ref{fig:Safety Analysis}]\\
\begin{figure}[htbp]
    \centering
    \includegraphics[width=0.25\textwidth,scale=1]{Imagenes_general/Gráfico_Capability.jpg}
    \includegraphics[width=0.20\textwidth,scale=1]{Imagenes_general/Tabla_Capability.png}
    \caption{Capability Analysis}
    \label{fig:Capability Analysis}
\end{figure}
\subsection{Capability}
ATP systems are limited in modernization due to reliance on established signaling and manual processes, while CBTC systems, supported by real-time sensors within a robust communication framework, excel in automated capacity, adaptability, and scalability to meet advancing railway demands. Figure [\ref{fig:Capability Analysis}]\\
\begin{figure}[htbp]
    \centering
        \includegraphics[width=0.27\textwidth,scale=1]{Imagenes_general/Gráfico_Maintenece.png}
    \includegraphics[width=0.15\textwidth,scale=1]{Imagenes_general/Tabla_Maintenece.jpg}
    \caption{Maintenance Analysis}
    \label{fig:Maintenance Analysis}
\end{figure}
\subsection{Maintenance}
Maintenance of ATP systems is challenged by aging components and fixed signaling, complicating updates, whereas CBTC systems benefit from a modern, modular design and software reliance, easing upgrades and adapting to new technologies with an emphasis on real-time communication. Figure [\ref{fig:Maintenance Analysis}]\\
\begin{figure}[htbp]
    \centering
        \includegraphics[width=0.23\textwidth,scale=1]{Imagenes_general/Gráfico_costs.png}
    \includegraphics[width=0.22\textwidth,scale=1]{Imagenes_general/tablas_costs.jpg}
    \caption{Costs Analysis}
    \label{fig:Costs Analysis}
\end{figure}
\subsection{Costs}
In railway systems cost analysis, ATP operates at lower costs due to simpler, existing technology and basic communication needs, while CBTC incurs higher costs owing to its use of advanced components and integrated communication systems, leading to complex cost assessments and greater overall expenses for implementation and future adaptability. [\ref{fig:Costs Analysis}]

The evaluation reveals that CBTC outperforms ATP across various critical parameters, notably demonstrating superior reliability, safety, and cost-effectiveness.[\ref{fig:Analysis}]

An analysis of automation grades was performed, taking into account the basic train operation functions in the Quito metro system with the current ATP, as well as the feasibility of conducting a study for the implementation of CBTC.

\subsection{UGT, Grade of  Automation }

In the table of the UNE$-$EN$-$62290$-$1 standard \cite{b17}, which addresses the basic functions for determining the grades of automation, the following is analyzed: the gray columns are not in effect to the current mass transportation system in Quito, the yellow column represents the grade of automation implemented in Quito's Metro, and the green columns represent the potential CBTC system that could be implemented in metropolitan urban transport.
\begin{table}[htbp]
\small
\centering
\caption{Grades of Automation (Quito's Metro)} 
\label{tab1}
\resizebox{9cm}{!}{
\begin{tabular}{
|>{\centering\arraybackslash}p{1.5cm}|
>{\centering\arraybackslash}p{3.5cm}|
>{\centering\arraybackslash\columncolor[gray]{0.8}}p{1.7cm}|
>{\centering\arraybackslash\columncolor[gray]{0.8}}p{1.7cm}|
>{\centering\arraybackslash\columncolor[RGB]{255, 255, 153}}p{1.9cm}|
>{\centering\arraybackslash\columncolor[RGB]{153, 255, 153}}p{1.7cm}|
>{\centering\arraybackslash\columncolor[RGB]{153, 255, 153}}p{1.7cm}|
}
\hline
\multicolumn{2}{|c|}{\multirow{2}{5cm}{Basic functions of train Operation}} & GoA0 & GoA1 & GoA2 & GoA3 & GoA4\\
\cline{3-7}
\multicolumn{2}{|c|}{} & On-Sight train operation & Non-automated train & Semiautomated train operation & Driverless train operation & Unatteneded train operation\\
\hline
\multirow{3}{1.5cm}{Ensure safe movement of train}
& Ensure safe route  
& \color{red}Human Intervention & Computer assisted & Computer assisted & Computer assisted & Computer assisted\\ 
& Ensure safe separation of train 
& \color{red}Human Intervention & Computer assisted & Computer assisted & Computer assisted & Computer assisted\\
& Ensure safe speed 
& \color{red}Human Intervention & \color{red} Human Intervention & Computer assisted & Computer assisted & Computer assisted\\
\hline
Drive train & Control acceletarion and Branking & \color{red}Human Intervention & \color{red} Human Intervention & Computer assisted & Computer assisted & Computer assisted\\
\hline
\multirow{2}{1.5cm}{Supervise guideway}& 
Prevent collison with obstacles 
& \color{red}Human Intervention & \color{red} Human Intervention & \color{red} Human Intervention & Computer assisted & Computer assisted\\
& Prevent collision with persons on tracks
& \color{red}Human Intervention & \color{red} Human Intervention & \color{red} Human Intervention & Computer assisted & Computer assisted\\
\hline
\multirow{3}{1.5cm}{Supervise passenger transfer}
& Control passengers doors 
&\color{red}Human Intervention & \color{red}Human Intervention & \color{red}Human Intervention & \color{red}Human Intervention & Computer assisted\\ 
& Prevent injuries to persons between on tracks
&\color{red}Human Intervention & \color{red}Human Intervention & \color{red}Human Intervention & \color{red}Human Intervention & Computer assisted\\
& Ensure safe starting conditions 
&\color{red}Human Intervention & \color{red}Human Intervention & \color{red}Human Intervention & \color{red}Human Intervention & Computer assisted\\
\hline
\multirow{2}{1.5cm}{Operate a train}& 
Put in or take out of operation 
& \color{red}Human Intervention & \color{red} Human Intervention & \color{red} Human Intervention & \color{red} Human Intervention & Computer assisted\\
& Supervise the status of the train
&\color{red}Human Intervention & \color{red} Human Intervention & \color{red} Human Intervention & \color{red} Human Intervention & Computer assisted\\
\hline
\multicolumn{2}{|c|}{\multirow{1}{5cm}{Ensure detection and management of emergency situacions } } &\color{red}Human Intervention & \color{red} Human Intervention & \color{red} Human Intervention & \color{red} Human Intervention & Computer assisted\\
\hline 
\multicolumn{3}{r}{\multirow{2}{*}{NOT APPLY$=$ GRAY}}&\multicolumn{2}{r}{\multirow{2}{*}{ATP$\_$UIO$=$ YELLOW}} &\multicolumn{2}{c}{\multirow{2}{*}{CBTC$\_$UIO$=$ GREEN}}
\end{tabular}
}
\end{table}
\subsection{Headway in Railway Signaling }
The term "headway" refers to the interval between consecutive trains on the same track, crucial for the capacity and safety of rail transport. CBTC (Communications-Based Train Control) systems significantly improve headway through continuous communication between trains and the control center, real-time adjustable virtual blocks, and Automatic Train Operation (ATO), optimizing frequency and capacity, as seen in the Beijing Metro, Paris Metro and Madrid Metro.\cite{b8}  In contrast, ATP (Automatic Train Protection) systems, like those in the Metro Quito and Metro Lima, have a higher and less efficient headway.\cite{b9} 
\subsubsection{Research}
This research was conducted on verifiable data of the current rolling stock, obtained from various official sources to simulate its dynamic behavior\cite{b11}, also incorporating data collected from the Quito Metro project. Considering the technical specifications of the trains, the railway system requirements\cite{b17}, and the regulations applicable to railway applications\cite{b16}, the necessary calculations were performed using Universidad Politécnica de Madrid software to achieve the simulation conditions for the analyzed project \cite{b4}.
\subsubsection{Conditions}
These conditions include fundamental parameters such as tractive effort, effective power, maximum braking force, and line operating speed. Tractive effort is the maximum force a train can apply to move and accelerate as shown in Equation:\\
Tractive effort \cite{b16}:
\begin{equation}
%\[
F_{\text{max,c}} = \frac{P_{c} \cdot 100 \cdot 3.6}{V}   
%\]
\label{ec:FT}
\end{equation}
\begin{itemize}
    \item $F_{\text{max,c}}$ is the continuous traction effort, 82.08 kN.
    \item $P_{c}$ is the continuous power of the motors, 570 kW. 
    \item $V$ is the speed of the train, 100 km/h.
\end{itemize}
Effective power, also represented in Equation \eqref{ec:FT}, denotes the engine's ability to maintain motion under varying load and resistance conditions. Maximum braking force indicates the train's capacity to decelerate safely and efficiently, while line operating speed defines the maximum speed at which trains can travel under normal service conditions.
\begin{table}[htbp]
\caption{Simulation cases}
\label{table:simulations}
\centering
\begin{tabular}{|c|c|c|c|c|}
%\toprule
\hline
\rowcolor[gray]{0.9}
N° & \begin{tabular}[c]{@{}c@{}}Station \\ stopping \\ time [sec]\end{tabular} & \begin{tabular}[c]{@{}c@{}}Headway \\ $[min]$\end{tabular} & \begin{tabular}[c]{@{}c@{}}N° \\ Trains\end{tabular} & Name\_Simulation \\ %\midrule
\hline
1  & 30 & 2   & 18 & \scriptsize SIM-1\_TIME-30\_HW-2\_TRAINS-18   \\ 
\hline
2  & 30 & 2   & 19 & \scriptsize SIM-2\_TIME-30\_HW-2\_TRAINS-19  \\
\hline
3  & 30 & 1.5 & 23 & \scriptsize SIM-3\_TIME-30\_HW-1.5\_TRAINS-23 \\
\hline
4  & 25 & 2   & 18 & \scriptsize SIM-4\_TIME-25\_HW-2\_TRAINS-18   \\
\hline
5  & 25 & 2   & 19 & \scriptsize SIM-5\_TIME-25\_HW-2\_TRAINS-19   \\
\hline
6  & 25 & 1.5 & 18 & \scriptsize SIM-6\_TIME-25\_HW-1.5\_TRAINS-18 \\
\hline
7  & 25 & 1.5 & 23 & \scriptsize SIM-7\_TIME-25\_HW-1.5\_TRAINS-23 \\
\hline
8  & 25 & 1.5 & 24 & \scriptsize SIM-8\_TIME-25\_HW-1.5\_TRAINS-24 \\
\hline
9  & 25 & 2   & 16 & \scriptsize SIM-9\_TIME-25\_HW-2\_TRAINS-16   \\
\hline
10 & 20 & 2   & 23 & \scriptsize SIM-10\_TIME-20\_HW-2\_TRAINS-23  \\ 
\hline 
\multicolumn{5}{c}{ }
\end{tabular}
\vspace{0.3cm}
\begin{minipage}{\linewidth} 
\footnotesize Note: Simulation cases presented in the table will be documented in a public repository \cite{b12}. This repository will include simulation scripts and test results, organized to facilitate navigation and access to the necessary information.
\end{minipage}
\end{table}

The initial simulation conditions used in the analyses are exhaustively described in Table [\ref{table:simulations}] and Table[\ref{table:train_data}]. These tables include critical variables such as station dwell time, headway, and the total number of trains in operation. These variables are essential for the detailed understanding and explanation of the current research. All other simulation cases will be published in a publicly accessible repository, allowing readers to compare and contrast additional cases not covered in this article.\cite{b12} 
\begin{table}[htbp]
\caption{Train Data}
\label{table:train_data}
\centering
\begin{tabular}{|p{4.5cm}|c|c|}
\hline
\rowcolor[gray]{0.9}
\centering Name & Value & \scriptsize Reference \\ 
\hline
Train Type & \scriptsize CAF-MQ117 & \cite{b10} \\
\hline
Maximum Acceleration & 1,25 [\scriptsize m/s$^2$]  & \cite{b11}  \\
\hline
Maximum Deceleration Service Brake \scriptsize(80\%)  & \centering 1,30 [\scriptsize m/s$^2$] & \cite{b11} \\
\hline
Maximum Wheel Power  & 570,00 [\scriptsize kW] & \cite{b12} \\
\hline
Maximum Braking Power  & 671,00 [\scriptsize kW] & \cite{b12} \\
\hline
Emergency Brake Application Time  & 1,50 [\scriptsize seconds] & \cite{b11} \\
\hline
Normal Deceleration & 0,80 [\scriptsize m/s$^2$]  & \cite{b11} \\
\hline
Emergency Brake Deceleration \scriptsize(100\%) & 1,15 [\scriptsize m/s$^2$]  & \cite{b11} \\
\hline
Maximum Traction Effort  & 82,08 [\scriptsize kN] & \cite{b12} \\
\hline
Maximum Service Speed & 100,00 [\scriptsize km/h] & \cite{b11} \\
\hline
Train Mass & 169,00 [\scriptsize ton]& \cite{b11} \\
\hline
Train Mass (Maximum Load) & 288,00 [\scriptsize ton] & \cite{b11} \\
\hline
Maximum Braking Effort & 78,55 [\scriptsize kN] & \cite{b12} \\
\hline
Train Length & 109,10 [\scriptsize m] & \cite{b11} \\
\hline
Track Gauge & 1435,00 \scriptsize[mm] & \cite{b11} \\
\hline
Frequency  & 60,00 \scriptsize[Hz] & \cite{b13} \\
\hline
Minimum Theoretical Headway  & 2 [\scriptsize min]& \cite{b14} \\
\hline
Number of Trains  & 18,00 [\scriptsize units] & \cite{b14} \\
\hline
Hold time at stations & 25,00 [\scriptsize seconds] & \cite{b15} \\
\hline
\end{tabular}
\end{table}

\subsubsection{SIM-4 TIME-25 HW-2 TRAINS-18}
Simulation 4, presents a scenario with 18 trains\cite{b20}, a 2-minute interval between trains\cite{b20}, and a 25-second stop at each station\cite{b15}. This simulation is adjusted to the currently implemented ATP system design. The time interval between the departure of the last train and the arrival of the first train in the opposite direction has a sufficient margin to allow a normal operation.\\
\begin{figure}[htbp]
    \centering
    \includegraphics[width=0.35\textwidth,scale=1]{Imagenes_general/SIM-4_TIME-25_HW-2_TRAINS-18.jpg}
    \caption{Simulation with 18 trains and headway 2 mins}
    \label{fig:Simulation 4: 18 Trains / 2 mins headway}
\end{figure}

\subsubsection{SIM-8 TIME-25 HW-1.5 TRAINS-24}
Simulation 8, considers an operational scenario with 24 trains, a 1.5-minute interval between trains, and a 25-second stop at each station\cite{b15}. The time from the start of the route of the last train on the left-to-right direction presents a headway greater than the theoretical established value compared to the start time of the route of the first train in the opposite direction.
\begin{figure}[htbp]
    \centering
    \includegraphics[width=0.35\textwidth,scale=1]{Imagenes_general/SIM-8_TIME-25_HW-1_5_TRAINS-24.jpg}
    \caption{Simulation with 24 trains and headway 1.5 mins}
    \label{fig:Simulation 8: 25 Trains / 1,5 mins headway}
\end{figure}

\subsubsection{SIM-9 TIME-25 HW-2 TRAINS-16}
Simulation 9, analyzes a scenario with 18 trains, of which 2 are under maintenance and 16 are in operation\cite{b20}. With a theoretical interval of 2 minutes between trains and a 25-second stop at each station, the simulation shows a lower train density. Under these conditions, the system can operate normally for a lower-than-expected passenger demand. This case reflects the current operational situation of the Quito Metro project\cite{b19}.
\begin{figure}[htbp]
    \centering
\includegraphics[width=0.35\textwidth,scale=1]{Imagenes_general/SIM-9_TIME-25_HW-2_TRAINS-16.jpg}
    \caption{Simulation with 16 trains and headway 2 mins}
    \label{fig:Simulation 9: 16 Trains / 2 mins headway}
\end{figure}

\subsection{Transition analysis of an ATP to a CBTC}
The transition from an ATP system to a CBTC system requires several steps and considerations to ensure capacity and safety. The key aspects of this transition are detailed below: Figure \ref{fig:Transition ATP - CBTC}
\begin{figure}[htbp]
    \centering
\includegraphics[width=0.45\textwidth,scale=1]{Imagenes_general/PROCESOS_TRANSITION.jpg}
    \caption{Transition ATP - CBTC}
    \label{fig:Transition ATP - CBTC}
\end{figure}
%%%%%%%%%%%%%%%%%%%%%%%%%%%%%%%%%%%%%%%%%%%%%%%%%%%%%%%%%%%%%%%%%%%%%%%%%%%%%%%%%%%%%%%%%%%%%%%%%%%%%%%%%%%%%%
%                                    RESULTADOS                                                              %
%%%%%%%%%%%%%%%%%%%%%%%%%%%%%%%%%%%%%%%%%%%%%%%%%%%%%%%%%%%%%%%%%%%%%%%%%%%%%%%%%%%%%%%%%%%%%%%%%%%%%%%%%%%%%%
\section{Results}
The importance level determined for the system analysis is as follows: reliability 30\%, safety 35\%, capacity 7\%, maintenance 10\%, and costs 18\%. The sum of these importance levels is 100\%. Figure \ref{fig:Analysis}
\begin{figure}[htbp]
    \centering
\includegraphics[width=0.4\textwidth, scale=1]{Imagenes_general/Grahp_general.png}
    \caption{ ATP and CBTC Analysis}
    \label{fig:Analysis}
\end{figure}
Different simulation cases are evaluated to determine how the system behaves in terms of headway and to optimize railway operation and traffic management with the possible implementation of CBTC in the study project as shown in Figure \ref{fig:Headway-Trains Analysis}. "Diff (min)" shows the difference in minutes between the simulated time for the last train to complete the outbound journey and the elapsed time for the first train on the return journey. "Trains Units" indicates the number of train units that could operate on the line. 
\begin{figure}[htbp]
    \centering
\includegraphics[width=0.4\textwidth,scale=1]{Imagenes_general/Analysis_Headway.jpg}
    \caption{Headway-Trains Analysis}
    \label{fig:Headway-Trains Analysis}
\end{figure}
The migration from ATP to CBTC improves the capacity and efficiency of the railway system through precise train localization and better traffic management, reduces the required physical infrastructure, enhances safety with automatic protection and continuous monitoring, and offers greater operational flexibility.\\

%%%%%%%%%%%%%%%%%%%%%%%%%%%%%%%%%%%%%%%%%%%%%%%%%%%%%%%%%%%%%%%%%%%%%%%%%%%%%%%%%%%%%%%%%%%%%%%%%%%%%%%%%%%%%
%                                CONCLUSIONES                                                               %
%%%%%%%%%%%%%%%%%%%%%%%%%%%%%%%%%%%%%%%%%%%%%%%%%%%%%%%%%%%%%%%%%%%%%%%%%%%%%%%%%%%%%%%%%%%%%%%%%%%%%%%%%%%%%

\section{Conclusions}
\begin{itemize}
\item The current capabilities of the railway signaling systems in Ecuador were selected for their technical complexity, impact on urban mobility, and potential for significant improvements. In the Quito Metro project, signaling is based on ATP, which includes continuous and punctual train supervision, although it still relies heavily on human intervention. The implementation of CBTC technologies alongside existing ATP could meet critical operational needs, as evidenced by the analysis results presented in Figure 15. A CBTC system outperforms ATP in terms of reliability (12. 85\%), safety (10. 5\%), capacity (2. 8\%), maintenance (1. 66\%) and costs (4. 5\%).

\item The adoption of the CBTC system in an environment with existing ATP in Ecuador significantly optimizes railway traffic management, increasing operational efficiency and line capacity. The CBTC reduces the interval between trains and increases line capacity, as demonstrated in case SIM-8, with 24 trains, 1.5-minute headway, and 25-second stops. Compared to case SIM-4, adjusted to the current implementation, the CBTC improves the headway by 30 seconds less than the theoretical value and increases the line by 6 trains. Surpassing the limitations of ATP, which, although it improves safety and efficiency with a GoA 2, relies more on human intervention, the CBTC allows greater operational efficiency through scalable degrees of automation and continuous communication up to the maximum GoA 4, as indicated in Table  [\ref{tab1}]

\item To overcome the technical obstacles in the implementation of the CBTC system in urban guided transport in Ecuador, it is recommended to conduct design studies, implementation processes, testing protocols, and commissioning as shown in Figure \ref{fig:Transition ATP - CBTC}. Additionally, develop cybersecurity protocols, project the use of artificial intelligence for predictive tasks, and apply current railway regulations. Financial obstacles will be managed through international credits, public-private partnerships, and primarily government subsidies, establishing a long-term financial plan for maintenance and updates. It is essential to conduct a cost-benefit analysis to maximize the return on investment and optimize resource allocation.
\end{itemize}
%%%%%%%%%%%%%%%%%%%%%%%%%%%%%%%%%%%%%%%%%%%%%%%%%%%%%%%%%%%%%%%%%%%%%%%%%%%%%%%%%%%%%%%%%%%%%%%%%%%%%%%%%%%%%%%%%
%                                    REFERENCIA BIBLIOGRAFÍCAS                                                  %
%%%%%%%%%%%%%%%%%%%%%%%%%%%%%%%%%%%%%%%%%%%%%%%%%%%%%%%%%%%%%%%%%%%%%%%%%%%%%%%%%%%%%%%%%%%%%%%%%%%%%%%%%%%%%%%%%

\begin{thebibliography}{00}
\bibitem{b1}IEEE Standard for Communications-Based Train Control (CBTC) Performance and Functional Requirements, IEEE Standard 1474.1, 25 February 2005.
\bibitem{b2}Ministerio de Transporte y Obras Públicas Ecuador,\textit{Proyecto "I001 Diagnóstico del Transporte Guiado a Nivel Nacional"},2021.
\bibitem{b3}Municipio de Cuenca,\textit{Especificaciones Técnicas de Obra Pública: Descripción General},2022.[Online]. Available en: \url{https://www.cuenca.gob.ec/system/files/4.11%20ESP.%20TEC.%201%20-%20DESCRIPCI%C3%93N%20GENERAL_0.pdf}
\bibitem{b4}Ar{\'a}nzazu Berbey Alvarez, Fernando Merchan, Jessica Guevara Cede{\~n}o, Alberto Cogley, and Rony Caballero,"Characterization of Line 1 of the Panama Metro," in \textit{Proceedings of the 2015 International Conference}, 2015. Figure 4. [Online]. Available: \href{https://api.semanticscholar.org/CorpusID:113501647}{https://api.semanticscholar.org/CorpusID:113501647}
\bibitem{b5}2023 IEEE/ACM International Conference on Computer Aided Design (ICCAD)," 2023. doi: \href{https://doi.org/10.1109/ICCAD57390.2023}{10.1109/ICCAD57390.2023}.
\bibitem{b6}Farnaz Towhidi, Arash Habibi Lashkari, and Raheleh Sadat Hosseini, "Binary Decision Diagram (BDD)," in \textit{2009 International Conference on Future Computer and Communication}, 2009, pp. 496-499. doi: \href{https://doi.org/10.1109/ICFCC.2009.31}{10.1109/ICFCC.2009.31}.
\bibitem{b7}Guang-Jun Jiang, Zong-Yuan Li, Guan Qiao, Hong-Xia Chen, Hai-Bin Li, and Hong-hua Sun, "Reliability Analysis of Dynamic Fault Tree Based on Binary Decision Diagrams for Explosive Vehicle," \textit{Mathematical Problems in Engineering}, vol. 2021, pp. 1-13, 2021. [Online]. Available: \url{https://api.semanticscholar.org/CorpusID:234833855}
\bibitem{b8} S. C. Kwok, H. K. Ho, L. Zhu, F. R. Yu, and F. Wang, \textit{Advances in Communications-Based Train Control Systems}, 1st ed. Boca Raton, FL: CRC Press, 2024, pp. 195-196.
\bibitem{b9}Línea 1 del Metro de Lima y Callao. "Portal Oficial de la Línea 1 del Metro de Lima y Callao". [Online].  Available: \url{https://www.lineauno.pe/}.[Accessed: 18-may-2024].
\bibitem{b10}CAF, "Proyecto Detalle: METRO QUITO", [Online]. Available: \url{https://www.caf.net/es/soluciones/proyectos/proyecto-detalle.php?p=284. Accessed: 18 de mayo de 2024}
\bibitem{b11}"LICITACIÓN PÚBLICA INTERNACIONAL RELI 01-2013 METRO DE QUITO-BID-CAF-BEI. PLIEGO DE PRESCRIPCIONES TÉCNICAS PARTICULARES," Metro de Quito, pp. 96, 160, 161.
\bibitem{b12} Autor:elsyv-rrivera \textit{Analyze-the-implementation-and-impact-of-CBTC-and-ATP}. Available: \url{ https://github.com/elsyv/Analyze-the-implementation-and-impact-of-CBTC-and-ATP}.[Accessed: 18-may-2024]
\bibitem{b13} "MEMORIA LICITACIÓN PÚBLICA INTERNACIONAL RELI 01-2013 METRO DE QUITO-BID-CAF-BEI, Ejecución de la primera línea del Metro de Quito, Fase 2: Construcción de las obras civiles y provisión y montaje del sistema de equipamiento e instalaciones, Componente instalaciones, Subsistema de señalización, A.T.P./A.T.O. vía-tren y A.T.S." Metro de Quito, 2019, pp. 22.
\bibitem{b14} "PRIMERA LINEA DEL METRO DE QUITO FASE II. MEMORIA DESCRIPTIVA DEL SISTEMA DE SEÑALIZACIÓN, Doc. N°: PMQ-CL1-D-SEN-GEN-GEN-MEM-6001-02D, Rev.: 02D" Metro de Quito, 2019, pp 15, 27.
\bibitem{b15} "Estructuración Técnica, Legal y Financiera del Contrato de Operación y Mantenimiento de la Primera Línea del Metro de Quito," Versión 04-190614, Documentación Interna Técnica, Jun. 2019, pp 14, 16.
\bibitem{b16} "Ministerio de Transportes, Movilidad y Agenda Urbana, Agencia Estatal de Seguridad Ferroviaria, Especificación Técnica de Circulación: Cálculo de Distancias de Frenado", 2020.
\bibitem{b17}UNE-EN 62290-1-2-3: International Electrotechnical Commission, "Railway Applications – Urban Guided Transport Management and Command$/$Control Systems" IEC 62290-1-2-3:2019
\bibitem{b18}"SISTEMA DE SEÑALIZACIÓN FERROVIARIA DE LA PRIMERA LÍNEA DE METRO DE QUITO, Doc. N°: $GEI_OM_DT_SEN_V1$, Metro de Quito, 2020, pp 5."
\bibitem{b19} Metro de Quito. "Portal Oficial de la Línea 1 del Metro de Quito". [Online]. Available: \url{https://metro-quito.com/incidente-en-la-estacion-quitumbe-del-metro-de-quito/}. [Accessed: 18-may-2024].
\bibitem{b20} Primicias. "Portal Oficial del medio digital Primicias". [Online]. Available:\url{https://www.primicias.ec/noticias/quito/metro-trenes-mantenimiento-danos-contrato/}. [Accessed: 18-may-2024].



\end{thebibliography}

\end{document}

%%%%%%%%%%%%%%%%%%%%%%%%%%%%%%%%%%%%%%%%%%%%%%%%%%%%%%%%%%%%%%%%%%%%%%%%%%%%%%%%%%%%%%%%%%%%%%%%%%%%%%%%%%%%%%%%%
%                                    FINAL DEL DOCUMENTO                                                        %
%%%%%%%%%%%%%%%%%%%%%%%%%%%%%%%%%%%%%%%%%%%%%%%%%%%%%%%%%%%%%%%%%%%%%%%%%%%%%%%%%%%%%%%%%%%%%%%%%%%%%%%%%%%%%%%%%